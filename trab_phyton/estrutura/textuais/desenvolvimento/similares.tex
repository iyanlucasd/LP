\chapter{LINGUAGENS SIMILARES OU CONFLITANTES}


Python é frequentemente comparado a outras linguagens interpretadas, como Java, JavaScript, Perl, Tcl, Smalltalk, C++, Common Lisp e Scheme\cite{comparacao}.
Na prática, a escolha de uma linguagem de programação é frequentemente ditada por outras restrições reais, como custo, disponibilidade, treinamento e investimento inicial e até mesmo compromisso emocional.

O subconjunto “baseado em objeto” do Python é equivalente ao JavaScript e, como o JS, o Python suporta um estilo de programação que usa funções e variáveis simples sem se envolver em definições de classe.
No entanto, para JavaScript, isso é tudo o que existe.

Espera-se que os programas Python sejam mais lentos do que os programas Java, porém levam menos tempo para serem desenvolvidos.
Os programas Python são tipicamente 3 a 5 vezes mais baixos do que programas Java comparáveis.
Essa diferença pode ser atribuída aos tipos de dados de alto nível integrados do Python e sua tipagem dinâmica.
Devido à digitação em tempo de execução, o tempo de execução do Python terá um desempenho melhor do que o Java.
Por exemplo, ao avaliar a expressão a + b, você deve primeiro examinar os objetos AEB para descobrir seu tipo, que é desconhecido em tempo de compilação.
Por essas razões, Python é muito mais adequado como uma linguagem "cola", enquanto Java é melhor caracterizado como uma linguagem de implementação de baixo nível.
Na verdade, os dois juntos formam uma ótima combinação.
Os componentes podem ser desenvolvidos em Java e combinados para formar aplicativos Python; Python também pode ser usado para prototipar componentes até que seu design possa ser "reforçado" em uma implementação Java.

Python brilha como uma linguagem de cola, usada para combinar componentes escritos em C++.
Como comparado em java, um programa tem de 3 a 5 vezes maior [nível], vemos essa diferença aumentar em relação de 5 a 10 vezes numa aplicação de C++ compara Python.

Por sua vez, o Tcl, que tradicionalmente armazena todos os dados como strings, é fraco em estruturas de dados e o código regular é executado muito mais devagar do que o Python.
Portanto, enquanto um grande aplicativo Tcl "típico" geralmente contém extensões Tcl escritas em C ou C ++ que são específicas para esse aplicativo, aplicativos Python equivalentes podem ser escritos em "Python puro".
Obviamente, o desenvolvimento puro em Python é muito mais rápido do que escrever e depurar um componente C ou C ++.
Python adotou a interface Tk como sua biblioteca padrão de componentes GUI.

Em Smalltalk, talvez a maior diferença entre eles seja a sintaxe "convencional" do Python, que oferece uma vantagem no treinamento de programadores.
No entanto, ele distingue os tipos de objeto embutidos das classes definidas pelo usuário e atualmente não permite a herança de tipos embutidos.
Python tem uma filosofia diferente em relação ao ambiente de desenvolvimento e distribuição de código. Enquanto Smalltalk tradicionalmente tem uma "imagem de sistema" monolítica, que inclui o ambiente e o programa do usuário, o Python armazena módulos padrão e módulos do usuário em arquivos individuais que podem ser facilmente reorganizados ou distribuídos fora do sistema.

Python e Perl vêm de um plano de fundo semelhante e possuem muitas características semelhantes, mas têm uma filosofia diferente.
Python enfatiza o suporte para metodologias de programação comuns, como desenho de estrutura de dados e programação orientada a objetos, e encoraja os programadores a escrever um código legível , fornecendo uma notação elegante, mas não excessivamente encriptada.

Em relação a common e lisp e Scheme, essas linguagens eram tão próximas dele em sua semântica dinâmica, mas tão diferentes em suas abordagens sintáticas que a comparação se tornou quase um argumento religioso: a falta de sintaxe de Lisp é vantagem ou desvantagem?
Deve-se notar que o Python possui recursos de introspecção semelhantes aos do Lisp e que os programas Python podem construir e executar fragmentos de programa em tempo real. Freqüentemente, as propriedades do mundo real são decisivas:
Common Lisp é grande e o mundo Scheme é fragmentado entre muitas versões incompatíveis, da qual Python tem uma implementação única, gratuita e compacta.

 
