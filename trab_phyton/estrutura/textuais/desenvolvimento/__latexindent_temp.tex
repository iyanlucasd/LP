% REVISÃO DE LITERATURA--------------------------------------------------------

\chapter{HISTÓRICO}
\label{chap:fundamentacaoTeorica}

Em 1989 o desenvolvimento do Python realmente teve início, nos primeiros meses de 1990 o autor já possuía uma versão mínima e operacional, pelo fim do ano de 1990 Python já era mais utilizada no CWI que a própria linguagem ABC.
\par No ano de 1991 Guido foi transferido do grupo Amoeba para o grupo Multimídia.
De acordo com o próprio Guido “ABC me deu a inspiração crucial para Python, o grupo Amoeba a motivação imediata e o grupo de multimídia fomentou seu crescimento". Ainda neste ano, no dia 20 de Fevereiro, foi lançada a primeira versão do Python, então denominada de v0.9.0. O anúncio foi feito no grupo de discussão (\textit{newsgroup}) alt.sources. A primeira release era composta de 21 partes uuencoded que juntos formavam um arquivo $.tar$. Velhos tempos…
\par Nesta primeira versão, o Python já contava com classes, herança, tratamento de exceções, funções, sistema de módulos (empresado da linguagem Modula-3) e os tipos de dado nativos list, dict, str, e etc.
\par Desde à primeira versão — e todas as outras versões lançadas dentro do CWI (Python 1.2) — possuíam uma licença derivada da licença MIT (na época utilizada pelo projeto X11), substituindo apenas a entidade legal responsável para "Stichting Mathematisch Centrum", organização pai do CWI. Abaixo um pequeno histórico de todas as versões lançadas no CWI:\footnote{
    essa é uma nota de rodapé
}

\begin{table}[!htb]
    \centering
    \caption[Histórico de versão]{Histórico de versão.
    \label{tab:Historico-versao}}
    \begin{tabular}{rrrrr}
        \toprule
        Mês/Ano & versão \\
        \midrule
        Fevereiro de 1991 & 0,77  \\
            0,19 & 0,74  \\
            1,00 & 1,00  \\
        \bottomrule
    \end{tabular}
    \fonte{\citeonline{Barbosa2004}}
\end{table}