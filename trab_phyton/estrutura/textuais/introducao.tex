% INTRODUÇÃO-------------------------------------------------------------------

\chapter{INTRODUÇÃO}

\label{chap:introducao}

Python é uma linguagem de programação de alto nível, interpretada de script, imperativa, orientada a objetos, funcional, de tipagem dinâmica e forte. 
Atualmente, possui um modelo de desenvolvimento comunitário, aberto e gerenciado pela organização sem fins lucrativos Python Software Foundation.
Apesar de várias partes da linguagem possuírem padrões e especificações formais, a linguagem, como um todo, não é formalmente especificada.
O padrão de facto é a implementação CPython.\footnote{
    CPython é a implementação principal da linguagem de programação Python, escrita em Linguagem C (mais especificamente o C89).
    É desenvolvida e mantida por Guido van Rossum e diversos outros desenvolvedores espalhados pelo mundo.
    O CPython é um interpretador de Bytecode. Ele possui uma interface funcional em diversas linguagens incluindo C, na qual os bindings podem ser escritos explicitamente em qualquer outra linguagem diferente de Python.\citeonline[p.~40]{pyMan}
}
 
Parte da cultura da linguagem gira ao redor de The Zen of Python, um poema que faz parte do documento "PEP 20 (The Zen of Python)", escrito pelo programador Tim Peters, descrevendo brevemente a filosofia do Python.
Pode-se vê-lo através de um \textit{easter egg} da linguagem pelo comando:
\\\textit{$>>>$ \textbf{import} this}
\begin{citacao}
    The Zen of Python, by Tim Peters

Beautiful is better than ugly.\\
Explicit is better than implicit.\\
Simple is better than complex.\\
Complex is better than complicated.\\
Flat is better than nested.\\
Sparse is better than dense.\\
Readability counts.\\
Special cases aren't special enough to break the rules.\\
Although practicality beats purity.\\
Errors should never pass silently.\\
Unless explicitly silenced.\\
In the face of ambiguity, refuse the temptation to guess.\\
There should be one-- and preferably only one --obvious way to do it.\\
Although that way may not be obvious at first unless you're Dutch.\\
Now is better than never.\\
Although never is often better than *right* now.\\
If the implementation is hard to explain, it's a bad idea.\\
If the implementation is easy to explain, it may be a good idea.\\
Namespaces are one honking great idea -- let's do more of those!\\

\end{citacao}